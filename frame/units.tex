\Chapter*{\tableOfUnitsName}
% Add table of units to the table of content. Keep this command here.
\addcontentsline{toc}{chapter}{\tableOfUnitsName}

\begin{longtblr}[
  theme=matchingCaptionNoContinuation,
  entry=none,
  label=none
  ]{
  colspec={Q[c,c]Q[c,l]X[c,c]},
  column{1}={leftsep=0.65cm, rightsep=1.65cm},
  rowsep=0pt
  }
  Formelzeichen            & Name                      & Einheit                              \\ \hline \hline
  \textit{l, s}            & Länge                     & \lbrack \unit{\metre}\rbrack         \\
  \textit{A}               & Fläche                    & \lbrack \unit{\metre\squared}\rbrack \\
  \textit{T}               & Sekunden                  & \lbrack \unit{\second}\rbrack        \\
  $\tau$                   & Zeitkonstante             & \lbrack \unit{\second}\rbrack        \\
  \textit{f}               & Frequenz                  & \lbrack \unit{\hertz}\rbrack         \\
  \textit{I}               & Strom                     & \lbrack \unit{\ampere}\rbrack        \\
  \textit{U}               & Spannung                  & \lbrack \unit{\volt}\rbrack          \\
  \textit{R}               & Widerstand                & \lbrack \unit{\ohm}\rbrack           \\
  \textit{C}               & Farad                     & \lbrack \unit{\farad}\rbrack         \\
  \textit{L}               & Induktivität              & \lbrack \unit{\henry}\rbrack         \\
  \textit{t} $(\vartheta)$ & Temperatur (Grad Celsius) & \lbrack \unit{\degreeCelsius}\rbrack \\
  %Beispiel
  %\textit{V}\textsubscript{OUT} & Ausgangsspannung des LM3489                    & \lbrack V\rbrack        \\
\end{longtblr}
