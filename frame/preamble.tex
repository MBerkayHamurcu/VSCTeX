% LTeX: enabled=false
\documentclass[
  12pt,
  a4paper,
  oneside,
]{report}

% Desired page format
\usepackage
[
  % showframe, % For debugging purposes
  driver=pdftex,
  a4paper,
  centering,
  left=2.5cm,
  right=2.5cm,
  top=3cm,
  bottom=3cm,
  headsep=1cm,
  headheight=21pt,
  footskip=1.5cm,
  marginparwidth=2cm
]{geometry}

% Encoding and font settings
\renewcommand{\baselinestretch}{1.5}
\usepackage[utf8]{inputenc}
\usepackage[T1]{fontenc}
\usepackage[ngerman]{babel}
\usepackage{lmodern}
\usepackage{csquotes}
\usepackage{parskip}
\DeclareTextCommand{\_}{T1}{%
  \leavevmode\kern.06em\vbox{\hrule width.6em}\kern.06em}

% Titles formatting
\usepackage[noindentafter]{titlesec}
\titleformat{\chapter}[hang]{\LARGE\bfseries\color{black}}{\thechapter\enskip}{0pt}{}
\titleformat{\section}[hang]{\Large\bfseries\color{black}}{\thesection\enskip}{0pt}{}
\titleformat{\subsection}[hang]{\large\bfseries\color{black}}{\thesubsection\enskip}{0pt}{}
\titleformat{\subsubsection}[hang]{\bfseries\color{black}}{\thesubsubsection\enskip}{0pt}{}
% Spaces after the titles
\titlespacing{\chapter}			{	0em}		{	-4ex}	{	2ex}

% Page header and footer
\usepackage{fancyhdr}

\newcommand{\headTextPaddingBottom}{\rule[-2ex]{0pt}{2ex}}

\fancypagestyle{plain}{%
  \fancyhf{}
  \fancyhead[L]{\headTextPaddingBottom\headingTopLeft}
  \fancyhead[C]{}
  \fancyhead[R]{\headTextPaddingBottom\small \nouppercase \leftmark}
  \fancyfoot[L]{\authorOne}
  \fancyfoot[C]{}
  \fancyfoot[R]{\large \thepage}
  %%	Lines in the header and footer
  \renewcommand{\headrulewidth}{0.4pt}
  \renewcommand{\footrulewidth}{0.4pt}
}

\fancypagestyle{empty}{%
  \fancyhf{}
  \fancyhead[L]{}
  \fancyhead[C]{}
  \fancyhead[R]{}
  \fancyfoot[L]{}
  \fancyfoot[C]{}
  \fancyfoot[R]{}
  \renewcommand{\headrulewidth}{0pt}
  \renewcommand{\footrulewidth}{0pt}
}

\fancypagestyle{appendix}{%
  \fancyhf{}
  \fancyhead[L]{\headTextPaddingBottom\headingTopLeft}
  \fancyhead[C]{}
  \fancyhead[R]{}
  \fancyfoot[L]{\authorOne}
  \fancyfoot[C]{}
  \fancyfoot[R]{\large \thepage}
  %%	Lines in the header and footer
  \renewcommand{\headrulewidth}{0.4pt}
  \renewcommand{\footrulewidth}{0.4pt}
}

\usepackage{xargs}

% Graphics support
\usepackage{graphicx}
\ifdefined\restrictedActive

  \usepackage{tikz}
\fi

\usepackage{needspace}

% Captions
\usepackage{subcaption}
\captionsetup{font=small,labelfont=bf,belowskip=-0.6cm}

% Citation
\usepackage[
  backend=biber,
  block=space,
  eprint=false,
  sorting=none,
  sortlocale=de_DE,
  url=false
]{biblatex}

% Integration of the .bib file
\addbibresource{literature.bib}
\BiblatexSplitbibDefernumbersWarningOff
\appto\bibfont{\setlength{\emergencystretch}{5pt}}

% Colors
\usepackage{xcolor}
% Custom colors used in listings
\definecolor{OliveGreen}{cmyk}{0.92,0,0.87,0.09}
\definecolor{Plum}{rgb}{0.72,0.16,0.72}
\definecolor{codegreen}{rgb}{0,0.6,0}
\definecolor{codegray}{rgb}{0.5,0.5,0.5}
\definecolor{codepurple}{rgb}{0.58,0,0.82}
\definecolor{backcolour}{rgb}{0.95,0.95,0.95}

\definecolor{tableContrast}{rgb}{0.93,0.93,0.94}

% Custom colors used with \colorbox
\definecolor{green1}{cmyk}{0.61, 0, 0.59, 0.38}
\definecolor{orange1}{cmyk}{0, 0.43, 0.94, 0.26}
% More transparent sustom colors used with \colorbox
\definecolor{green2}{cmyk}{0.51, 0, 0.49, 0.28}
\definecolor{orange2}{cmyk}{0, 0.33, 0.84, 0.16}
\definecolor{bluebright}{cmyk}{1, 0.17, 0, 0.20}
\definecolor{yellowdark}{cmyk}{0, 0.16, 1, 0.24}

% Code block support
\usepackage{listings}
\lstdefinestyle{customCStyle}{%
  backgroundcolor=\color{backcolour},
  commentstyle=\color{codegreen},
  keywordstyle=\color{magenta},
  numberstyle=\tiny\color{codegray},
  stringstyle=\color{codepurple},
  basicstyle=\ttfamily\scriptsize\linespread{0.65},
  breakatwhitespace=false,
  breaklines=true,
  captionpos=t,
  keepspaces=true,
  numbers=left,
  numbersep=5pt,
  showspaces=false,
  showstringspaces=false,
  showtabs=false,
  tabsize=2,
  xleftmargin=.03\textwidth,
  xrightmargin=.03\textwidth,
  belowcaptionskip=.4\baselineskip,
  language=C
}

% Import the document variables
% Beware: Commenting out, changing or removing lines is only allowed where explicitly stated!

% Set document variables
\newcommand{\documentLanguage}{de} % or f.i. en

\newcommand{\projectTitle}{EDIT\_ME\_TITEL}
\newcommand{\projectType}{EDIT\_ME\_Arbeitsart}
\newcommand{\authorOne}{EDIT\_ME\_Berkay Hamurcu}
\newcommand{\submissionDate}{EDIT\_ME\_Datum}
\newcommand{\projectTimeframe}{EDIT\_ME\_Bearbetungsdauer}
\newcommand{\matriculationNumber}{EDIT\_ME\_Matrikelnummer}
\newcommand{\course}{EDIT\_ME\_Kurs}
\newcommand{\company}{EDIT\_ME\_Firma}
\newcommand{\companyLocation}{EDIT\_ME\_Standort}
\newcommand{\tutor}{EDIT\_ME\_B. Eng., xyz}
\newcommand{\secondTutor}{EDIT\_ME\_M. Eng., abc}
\newcommand{\studyProgram}{EDIT\_ME\_Studiengang}
\newcommand{\dhbw}{EDIT\_ME\_Ort}
\newcommand{\headingTopLeft}{EDIT\_ME\_Kopfzeile}

% Set heading and misc names
\renewcommand*\contentsname{Inhaltsverzeichnis}
\renewcommand*\listfigurename{Abbildungsverzeichnis}
\renewcommand*\figurename{Abbildung}
\renewcommand*\listtablename{Tabellenverzeichnis}
\renewcommand*\tablename{Tabelle}
\newcommand{\tableOfAcronymsName}{Abkürzungsverzeichnis}
\renewcommand*\lstlistlistingname{Quellcodeverzeichnis}
\renewcommand*\lstlistingname{Quellcode}
\newcommand{\equationName}{Formel}
\newcommand{\listOfEquationsName}{Formelverzeichnis}
\newcommand{\userFrameName}{Rahmen}
\newcommand{\listOfUserFramesName}{Rahmenverzeichnis}
\definecolor{userFrameColor}{RGB}{218, 219, 243}
\newcommand{\bibliographyName}{Literaturverzeichnis}
\newcommand{\printMediaTitle}{Printquellen}
\newcommand{\webMediaTitle}{Internetquellen}
\newcommand{\BIBand}{und}
\newcommand{\appendixName}{Anhang}
\newcommand{\toDosName}{To-Do's}

\newcommand{\mainLogo}{media/DHBW_Logo.png}
\newcommand{\confidentialText}{Vertraulich}
\newcommand{\projectTimeframeText}{Bearbeitungszeitraum}
\newcommand{\matriculationNumberText}{Matrikelnummer}
\newcommand{\courseText}{Kurs}
\newcommand{\companyText}{Ausbildungsfirma}
\newcommand{\tutorText}{Betreuer}
\newcommand{\secondTutorText}{Zweitbetreuer}
\newcommand{\tableContinuationText}{Fortsetzung auf der nächsten Seite}
\newcommand{\tableContinuationCaptionText}{(Fortsetzung)}
\newcommand{\watermarkText}{Wasserzeichen}

% Custom bib classification
\defbibfilter{printMediaFilter}{
  type=book or type=article
}
\defbibfilter{webMediaFilter}{
  not type=book and not type=article
}

% Enable/Disable an option by activating/commenting the according line:
\newcommand{\listofunitsVisible}{}
\newcommand{\listofacronymsVisible}{}
\newcommand{\listoffiguresVisible}{}
\newcommand{\listoftablesVisible}{}
\newcommand{\listoflistedEquationsVisible}{}
\newcommand{\listoflistingsVisible}{}
\newcommand{\listofuserFramesVisible}{}
% \newcommand{\secondLogo}{media/DHBWEngineering_Logo.pdf}
% \newcommand{\watermarkActive}{}
% \newcommand{\restrictedActive}{}
\newcommand{\toDosActive}{}


% Maths
\usepackage{amsmath}

% Prepare list of equations
\DeclareCaptionType{listedEquation}[\equationName][\listOfEquationsName]

% Define custom equations macro
\newcommand{\Equation}[2]{%
  \begin{listedEquation}[H]
    {#2}

    \caption{#1}
    \label{#1}
  \end{listedEquation}
}

\usepackage{framed}

% Prepare list of frames
\DeclareCaptionType{userFrame}[\userFrameName][\listOfUserFramesName]

% Define custom frame environment
\renewcommand{\FrameCommand}{\fcolorbox{black}{userFrameColor}}
\newenvironment{Frame}[1][ARGUNDEFINED]{
  \captionsetup{type=figure,belowskip=0cm}

  \vspace{3pt}

  \ifstrequal{#1}{ARGUNDEFINED}{%
  }{

    \needspace{2\baselineskip}
    \captionof{userFrame}{#1}
    \label{#1}
  }

  \setlength{\fboxsep}{20pt}

  \begin{framed}
    }{
  \end{framed}

  \vspace{3pt}
}

% Define custom bullet list macro
\newenvironment{Itemize}
{ \begin{itemize}
    \setlength{\itemsep}{0pt}
    \setlength{\parskip}{0pt}
    \setlength{\parsep}{0pt}     }
    { \end{itemize}                  }

% Tables
% tabularray, the best package
% for tables with every feature
% one could wish for. Easy and
% clean to use
\usepackage{tabularray}

% Set tabularray theme
\NewTblrTheme{matchingCaption}{
  \SetTblrStyle{caption-tag}{font=\small\bfseries}
  \SetTblrStyle{caption-text}{font=\small\mdseries}
  %For centering the text of continuation footer
  \SetTblrStyle{contfoot}{font=\small\mdseries, halign=\tableContinuationTextPos}
}

% Adapt the tabularray continuation texts
\DefTblrTemplate{contfoot-text}{contfootLanguageAdaption}{\tableContinuationText}
\SetTblrTemplate{contfoot-text}{contfootLanguageAdaption}
\DefTblrTemplate{conthead-text}{contfootLanguageAdaption}{\tableContinuationCaptionText}
\SetTblrTemplate{conthead-text}{contfootLanguageAdaption}

% Acronyms support
\usepackage{array}
\usepackage{longtable}
\usepackage{acro}[=v3]

\acsetup{%
make-links,
%link-only-first,
templates/colspec={>{\bfseries{\hspace{-16pt}}}l@{\hspace{1.9cm}}p{0.7\linewidth}}
}

% Watermark support
\usepackage{eso-pic}

% To-Do notes support
\newcommand{\toDosDisabler}{disable}
\ifdefined\toDosActive

  \renewcommand{\toDosDisabler}{}
\fi

\usepackage[colorinlistoftodos,textsize=tiny,ngerman,\toDosDisabler]{todonotes}
\newcommandx{\unsure}[2][1=]{\todo[linecolor=red,backgroundcolor=red!25,bordercolor=red,#1]{#2}}
\newcommandx{\change}[2][1=]{\todo[linecolor=blue,backgroundcolor=blue!25,bordercolor=blue,#1]{#2}}
\newcommandx{\info}[2][1=]{\todo[linecolor=OliveGreen,backgroundcolor=OliveGreen!25,bordercolor=OliveGreen,#1]{#2}}
\newcommandx{\improvement}[2][1=]{\todo[linecolor=Plum,backgroundcolor=Plum!25,bordercolor=Plum,#1]{#2}}
\newcommandx{\thiswillnotshow}[2][1=]{\todo[disable,#1]{#2}}

% Mainly for `H` alignment support
\usepackage{float}

% Optional watermark
\ifdefined\watermarkActive
  %
  \AddToShipoutPicture{\AtTextCenter{%
      \makebox(10,20)[c]{\resizebox{0.8\textwidth}{!}{%
          \rotatebox{45}{\textsf{\textbf{\color{lightgray}\watermarkText}}}}}}}
  %
\fi

% Automatic includeonly management
\ifdefined\fullCompile

  % Compile all \include macros if not in fast mode
\else

  % This file is autogenerated. Do not edit it manually. Changes will be lost.

\includeonly{}
% lastChanges{}

\fi

\makeatletter

% Define custom chapter macros
\newcommand{\Chapter}{\@ifstar{\@ChapterStar}{\@ChapterNoStar}}

% \Chapter*
\newcommand{\@ChapterStar}{\@ifnextchar[{\@optStarChapter}{\@noOptStarChapter}}

% \Chapter
\newcommand{\@ChapterNoStar}{\@ifnextchar[{\@optChapter}{\@noOptChapter}}

% No optional argument given
\newcommand{\@noOptStarChapter}[1]{%
  \chapter*{#1} % Create numbered chapter with title
  \ifnum\thechapter>0

    \addcontentsline{toc}{chapter}{#1}
  \fi
  \markboth{#1}{#1} % Set header to title
  \label{#1}
}
\newcommand{\@noOptChapter}[1]{%
  \chapter{#1}
  \markboth{\thechapter\space #1}{\thechapter\space #1}
  \label{#1}}

% Optional argument is given
\newcommand{\@optStarChapter}{\@dblarg\@largOptStarChapter}
\newcommand{\@largOptStarChapter}[2][]{%
  % Create numbered chapter with short title in the toc
  % Optional arg: #1, mandatory arg: #2
  \chapter*[#1]{#2}
  \ifnum\thechapter>0

    \addcontentsline{toc}{chapter}{#1}
  \fi
  \markboth{#1}{#1} % Set header to short title
  \label{#1}
}
\newcommand{\@optChapter}{\@dblarg\@largOptChapter}
\newcommand{\@largOptChapter}[2][]{%
  \chapter[#1]{#2}
  \markboth{\thechapter\space #1}{\thechapter\space #1}
  \label{#1}
}

% Add table of content entries for the lists
\g@addto@macro\listoffigures{%
  \addtocontents{lof}{\protect\renewcommand*{\protect\@pnumwidth}{4.0em}}
  % Add list of figures to the table of content
  \addcontentsline{toc}{chapter}{\listfigurename}
}

\g@addto@macro\listoftables{%
  \addtocontents{lot}{\protect\renewcommand*{\protect\@pnumwidth}{4.0em}}
  % Add list of tables to the table of content
  \addcontentsline{toc}{chapter}{\listtablename}
}

\g@addto@macro\lstlistoflistings{%
  \addtocontents{lol}{\protect\renewcommand*{\protect\@pnumwidth}{4.0em}}
  % Add list of code blocks to the table of content
  \addcontentsline{toc}{chapter}{\lstlistlistingname}
}

\g@addto@macro\printbibheading{%
  % Add the bibliography to the table of content
  \addcontentsline{toc}{chapter}{\bibname}
}

% Customize the abstract environment
\renewenvironment{abstract}{
  \newpage
  \if@twocolumn

    \section*{\abstractname}%
  \else

    \small
    \begin{flushleft}%
      {\bfseries \abstractname\vspace{-.5em}\vspace{\z@}}%
    \end{flushleft}%
  \fi}
{\if@twocolumn\else\endflushleft\fi}

\makeatother

% Text bullets for the abstract
\newcommand{\kwtextbullet}{\hspace{-0.05cm}\textbullet\hspace{0.08cm}}

% Prepare appendix
\newcommand{\beginAppendix}{%
  \newpage

  \setcounter{chapter}{0}
  % Switch to roman page numbers in the appendix and
  % increase the width of the box for those page numbers
  \newcounter{svpage}
  \setcounter{svpage}{\value{page}}
  \pagenumbering{Roman}
  \setcounter{page}{\value{svpage}}
  \addtocontents{toc}{\protect\renewcommand*{\protect\@pnumwidth}{4.0em}}
}

\newcommand{\Appendix}{%
  \ifdefined\fullCompile

    \thispagestyle{appendix}
    \vspace*{\fill}
    \begin{center}
      \Huge \textbf{\appendixName}
    \end{center}
    \vspace*{\fill}
    \vspace*{1.5cm}

    \newpage

    \appendix
  \fi
}

% PDF settings
\usepackage[%
  pdftitle={\projectTitle},
  pdfauthor={\authorOne},
  pdfsubject={\projectTitle},
  pdfcreator={latexmk},
  pdfpagemode=UseOutlines, % Show the table of contents in the sidepanel by default 
  pdfdisplaydoctitle=true, % Show the document title instead of the file name
  pdflang={\documentLanguage},
  bookmarksnumbered=true,
  plainpages=false,
]{hyperref} % Loading hyperref as the last package

\usepackage{url}
