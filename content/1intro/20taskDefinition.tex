\Chapter{Aufgabenstellung}
\label{ch:Aufgabenstellung}

\Equation{Boolesche Ungleichung \ref{ch:Aufgabenstellung}}{Boolesche Ungleichung}{
  \begin{equation}
    P(\bigcup_{n=1}^n A_n) \leq \sum_{n=1}^n P(A_n)
  \end{equation}
}

Wenn die Ereignisse $A_n$ disjunkt sind, dann wird die Ungleichung
in \autoref{Boolesche Ungleichung} zu einer Gleichheit. \unsure{Ist das richtig?}

\begin{equation}\label{pythTheorem}
  a^2+b^2=c^2
\end{equation}

Unabhängig davon definiert \autoref{pythTheorem} die Länge der Seiten eines rechtwinkligen Dreiecks, wobei $c$ die Hypotenuse darstellt. \unsure{Und das auch?}
