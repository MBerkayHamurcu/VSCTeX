\Chapter{Umsetzung des Testmodells}

Hauptteil\ldots

\section{Hauptteil}
Inline Maths: $t_{sim} >> t_{real}$

Zwei Bilder vertikal geteilt. Das Format der beiden Bilder zueinander ist relevant, die subfigure breiten
müssen dementsprechend angepasst werden!:
\begin{figure}[H]
  \hfill
  \begin{subfigure}[t]{0.65\textwidth}
    \includegraphics[width=1\textwidth]{media/exampleA1.jpg}
  \end{subfigure}
  \hfill
  \begin{subfigure}[t]{0.3\textwidth}
    \includegraphics[width=1\textwidth]{media/exampleA2.jpg}
  \end{subfigure}
  \hfill
  \captionsetup{width=0.8\textwidth}
  \caption[Projektierung]{Projektierung einer Antriebseinheit}
  \label{antrieb1}
  \vspace{-0.4cm}
\end{figure}

\newpage

\subsection{Unterkapitel vier Bilder}

\improvement[inline]{The following section needs to be rewritten!}
Vier Bilder vertikal und horizontal geteilt. Das Format der beiden Bilder zueinander ist relevant, die subfigure breiten
müssen dementsprechend angepasst werden!:
\begin{figure}[H]
  \hfill
  \begin{subfigure}[t]{0.49\textwidth}
    \includegraphics[width=1\textwidth]{media/RED.jpg}
  \end{subfigure}
  \hfill
  \begin{subfigure}[t]{0.49\textwidth}
    \includegraphics[width=1\textwidth]{media/RED.jpg}
  \end{subfigure}
  \hfill

  \vspace{0.01\textwidth}

  \hfill
  \begin{subfigure}[b]{0.49\textwidth}
    \includegraphics[width=1\textwidth]{media/RED.jpg}
  \end{subfigure}
  \hfill
  \begin{subfigure}[b]{0.49\textwidth}
    \missingfigure[figwidth=7.7cm,figcolor=white]{Vergleich des Aufwands}
  \end{subfigure}
  \hfill

  \captionsetup{width=0.8\textwidth}
  \caption[Vergleich Arbeitsspeicherbelegung - Rechenoperationen]{Vergleich Simulationsaufwand (Arbeitsspeicherbelegung) - Menge der Rechenoperationen}
  \label{vergleich1dot5}
  \vspace{-0.4cm}
\end{figure}
Tabellenbeispiele \ref{erg2-ref} und \ref{erg1-rech}:

\begin{longtblr}[
  theme=matchingCaption,
  caption={Testergebnis - Referenzprojekt (2)},
  entry={Referenzprojekt (2)},
  label={erg2-ref}
  ]{
  colspec={Q[c,m]Q[c,m]},
  rowsep=2pt,
  hlines={black, 0.5pt},
  vlines={black, 0.5pt},
  rowhead=1,
  row{odd}={tableContrast},
  row{1}={blue9,font=\bfseries}
  }
  Messwert                      & Durchschnitt \\
  Prozessorauslastung (\%)      & 29,6         \\
  \ac{sps}-Zykluszeit (ms)      & 1,0154       \\
  Simulationslast (\%)          & 5,8          \\
  Simulationslast okay (\%)     & 99,98        \\
  Simulationslast kritisch (\%) & 0,016        \\
  Simulationslast überlast (\%) & 0,004        \\
  Prozessorauslastung (\%)      & 29,6         \\
  Weitere Einträge              & 1,0154       \\
  Weitere Einträge              & 9,8          \\
  Weitere Einträge              & 9,1          \\
\end{longtblr}

\begin{longtblr}[
  theme=matchingCaption,
  caption={Testergebnis - Rechenoperationen (1)},
  entry={Rechenoperationen (1)},
  label={erg1-rech}
  ]{
  width=1\linewidth,
  colspec={Q[1,c,m]Q[1,c,m]Q[1,c,m]Q[1,c,m]},
  rowsep=2pt,
  hlines={black, 0.5pt},
  vlines={black, 0.5pt},
  rowhead=1,
  row{odd}={tableContrast},
  row{1}={blue9,font=\bfseries}
  }
  Instanz                               & Privilegierte \ac{cpu} Zeit (\%) & Working Set MB & Private Bytes MB \\
  Siemens Automation. Portal            & 17,2                             & 3823           & 4379             \\
  Siemens. Simatic-Simulation. Instance & 6,1                              & 707            & 2143             \\
  SIMIT-CS                              & 77,5                             & 163            & 39               \\
\end{longtblr}
