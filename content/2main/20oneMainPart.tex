\Chapter[Spezifikation Kurztitel]{Spezifikation des Testmodells}

Einführung im \autoref{Spezifikation Kurztitel}\ldots

\section{Auflistung Beispiel}

\begin{Itemize}
  \item Formulierung quantitativer Leistungskennzahlen
  \newline
  Definition quantitativer KPIs ist von Bedeutung, um die Leistungsfähigkeit der
  verschiedenen Simulationssystem-Architekturen objektiv zu bewerten. Dieses Teilziel
  beinhaltet die Identifizierung geeigneter KPIs und die Entwicklung von Metriken zur Messung
  der Simulationskapazität
  \item Aufbau des Testmodells
  \newline
  Die Entwicklung des konfigurierbaren Testmodells bildet die Grundlage für die Durchführung von
  Tests an verschiedenen Simulationssystem-Architekturen. Hierbei gilt es, die Plattform so zu gestalten,
  dass sie flexibel an verschiedene Simulationsszenarien anpassbar ist
  \item Entwicklung eines flexiblen Konfigurationssystems\improvement{This really needs to be improved! What was I thinking?!}
  \newline
  Die Implementierung eines flexiblen Konfigurationssystems ermöglicht die Anpassung des Testmodells an
  unterschiedliche Simulationskapazitäten. Hierbei müssen Parameter definiert werden, die es dem Testmodell
  ermöglichen, den Simulationsaufwand je nach Architektur anzupassen
\end{Itemize}

\section{Mindestanforderungen}

Mit Inhalten, die mithilfe von Textboxen \colorbox{yellowdark}{markiert} werden sollen \ldots
